\documentclass[11pt]{article}
\usepackage[pdftex]{graphicx}

\setlength{\textwidth}{6.5in}
\setlength{\textheight}{9.25in}

\setlength{\voffset}{-0.0in}
\hoffset=-50pt
\topmargin=0pt
\headheight=0pt
\headsep=0pt
\evensidemargin=72pt % 1in+7pt

\setlength{\parskip}{0.5ex}
\setlength{\parindent}{1em}
\setlength{\floatsep}{1pt}
\setlength{\textfloatsep}{2pt}
\setlength{\intextsep}{2pt}

\begin{document}

\title{Map{R}educe in {MPI} for Large-scale Graph Algorithms}

\author{
Steven J.~Plimpton, Karen D.~Devine, Jon?, Cohen?, Others? \\
Sandia National Laboratories \\
Albuquerque, NM \\
sjplimp@sandia.gov
}

\date{}

\maketitle

\centerline{Keywords: MapReduce, message-passing, MPI, graph
algorithms, RMAT matrices}

\vspace*{0.4in}

\begin{abstract}

We describe a library that allows parallel algorithms to be expressed
in the MapReduce paradigm, where the user does not need to write
explicit parallel code, but instead provides a collection of ``map''
and ``reduce'' functions which operate on portions of a data set.  The
library itself uses MPI message passing calls to perform the needed
parallel operations across distributed memory.  We highlight how
MapReduce operations work in an MPI context, and also how they can be
coded in an out-of-core manner for data sets that do not fit within
the aggregate memory of a parallel machine.  Our motivation for
creating this library was to enable graph algorithms to be coded as
MapReduce operations, allowing processing of Terabyte-scale data sets.
We outline a MapReduce version of several such algorithms: vertex
ranking via PageRank, connected component identification, triangle
finding, Luby's algorithm for maximally independent sets, and
single-source shortest-path.  To input arbitrarily large artificial
graphs we generate RMAT matrices in parallel; we also provide a
MapReduce version of this operation.  Performance and scalability
results for the various algorithms are presented for graphs of varying
sizes on a distributed-memory cluster.  For some cases, we compare our
results with timings for different algorithms, different machines, and
different MapReduce software, namely Hadoop.  Our open-source library
is written in C++, is callable from C++, C, Fortran, or scripting
languages such as Python, and can run on any parallel platform that
supports MPI.

\end{abstract}

\pagebreak

\section{Introduction}
\label{sec:intro}

MapReduce is the programming paradigm popularized by Google
researchers Dean and Ghemawat \cite{Dean}.  Their motivation was to
enable rapid development and deployment of analysis programs to
operate on massive data sets residing on Google's large distributed
clusters.  They introduced a novel way of thinking about certain kinds
of large-scale computations as a ``map'' operation followed by a
``reduce'' operation.  The power of the paradigm is that when cast in
this way, a nominally serial algorithm now becomes two highly parallel
operations working on data local to each processor, sandwiched around
an intermediate data-shuffling operation that requires inter-processor
communication.  The user need only write serial code for the
application-specific map and reduce functions; the parallel data
shuffle can be encapsulated in a library since its operation is
independent of the application.

The Google implementation of MapReduce is a C++ library with
communication between networked machines via remote procedure calls.
It allows for fault tolerance when large numbers of machines are used,
and can use disks as out-of-core memory to process petabyte-scale data
sets.  Tens of thousands of MapReduce programs have since been written
by Google researchers and are a significant part of the daily compute
tasks run by the company \cite{Dean2}.

Similarly, the open-source Hadoop implementation of MapReduce
\cite{Hadoop}, has become widely popular in the past few years for
parallel analysis of large-scale data sets at Yahoo and other
data-centric companies, as well as in university and laboratory
research groups, due to its free availability.  MapReduce programs in
Hadoop are typically written in Java, though it also supports use of
stand-alone map and reduce kernels, which can be written as shell
scripts or in other languages.

More recently, MapReduce formulations of traditional number-crunching
kinds of scientific computational tasks have been described, such as
post-processing analysis of simulation data \cite{Tu}, graph
algorithmics \cite{Cohen}, and linear algebra operations \cite{Fox}.
The paper by Tu et al \cite{Tu} was particularly insightful to us,
because it described how MapReduce could be implemented on top of the
ubiquitous distributed-memory message-passing interface (MPI), and how
the intermediate data-shuffle operation is conceptually identical to
the familiar MPI\_Alltoall operation.  Their implementation of
MapReduce was within a Python wrapper to simplify the writing of user
programs.  The paper motivated us to develop our own C++ library built
on top of MPI for use in graph analytics, which we initially released
as open-source software in mid-2009 \cite{MRMPI}.  We have since
worked to optimize several of the library's underlying algorithms and
to enable its operation in out-of-core mode on larger data sets.
These algorithmic improvements are described in this paper and are
part of the current downloadable version \cite{MRMPI}.

The MapReduce-MPI (MR-MPI) library described in this paper is a
simple, lightweight implementation of basic MapReduce functionality,
with the following features and limitations:

\begin{itemize}

\item {\it C++ library using MPI for inter-processor communication:}
The user writes a (typically) simple main program which runs on each
processor of a parallel machine, making calls to the MR-MPI library.
For map and reduce operations, the library calls back to user-provided
map() and reduce() functions.  The use of C++ allows precise control
over the memory and format of data allocated by each processor during
a MapReduce.  Library calls for performing a map, reduce, or data
shuffle, are synchronous, meaning all the processors participate and
finish the operation before proceeding.  Similarly, the use of MPI
within the library is the traditional mode of MPI\_Send and
MPI\_Recv calls between processor pairs using large aggregated
messages to improve bandwidth performance and reduce latency costs.  A
recent paper by \cite{Dongarra} also outlines the MapReduce formalism
from an MPI perspective, though they advocate a more asyncrhonous
approach, using one-way communication of small messages.

\item {Small, portable:} The entire MR-MPI library is a few thousand
lines of standard C++ code.  For parallel operation, the program is
linked with MPI, a standard message passing library available on all
distributed memory machines.  For serial operation, a dummy MPI
library (provided) can be substituted.  As a library, it can be
embedded in other codes \cite{Titan,TitanURL} to enable them to
perform MapReduces.

\item {\it In-core or out-of-core operation:} Each MapReduce object that a
processor defines allocates per-processor ``pages'' of memory, where
the page size is determined by the user.  Typical MapReduce operations
can be performed using a few such pages.  If the data set fits in a
single page (per processor), then the library performs its operations
in-core.  If the data set exceeds the page size, then processors each
write to temporary disk files (to local disk or a parallel file
system) as needed and subsequently read from them.  This allows
processing of data sets larger than the aggregate memory of all the
processors, i.e. up to the available aggregate disk space.

\item {\it Flexible programmability:} An advantage of writing a
MapReduce program on top of MPI, is that the user program can invoke
MPI calls directly, if desired.  For example, one-line calls to
MPI\_Allreduce are often useful in determining the status of an
iterative graph algorithm, as described in Section \ref{sec:graph}.
The library interface also provides a user data pointer as an argument
passed back to all callback functions, so it is easy for the user
program to store ``state'' on each processor, accessible during the
map and reduce operations.  For example, various flags can be stored
that alter the operation of a map or reduce function, as can richer
data structures, that accumulate the results.

\item {\it C++, C, and Python interfaces:} A C++ interface to the
MR-MPI library means a user program instantiates and then invokes
methods in one or more MapReduce objects.  A C interface means the
library can also be called from C or other high-level languages such as
Fortran.  A C interface also means the library can be easily wrapped
by Python via the Python ``ctypes'' module.  The library can then be
called from a Python script, allowing the user to write map() and
reduce() callback functions in Python.  If a machine supports running
Python in parallel, a parallel MapReduce can also be run in this mode.

\item {No fault tolerance:} Current MPI implementations do not enable
easy detection of a dead processor, or retrieval of the data it was
working on.  So like most MPI programs, a parallel program calling the
MR-MPI library will hang or crash if a processor goes away.  Unlike
Hadoop, and its HDFS file system which provides for data redundancy,
the MR-MPI library simply reads and writes simple, flat files.  It can
use local per-processor disks, or a parallel file system, if
available, but these typically provide no data redundancy.

\end{itemize}

The remainder of the paper is organized as follows.  The next two
sections \ref{sec:mr} and \ref{sec:outcore} describe how in-core and
out-of-core MapReduce primitives are formulated as MPI-based
operations in the MR-MPI library.  Section \ref{sec:graph} briefly
describes the formulation of several common graph algorithms as
MapReduce operations.  Section \ref{sec:results} gives performance
results for these algorithms running on a parallel cluster for graphs
%KDDKDD CHECK SIZES
ranging in size from 1 million to 1 trillion vertices or edges.  In
this section, we highlight the performance and complexity trade-offs
of a MapReduce approach versus other more special-purpose algorithms.
The latter generally perform better but are harder to implement
efficiently on distributed memory machines, due to the required
explicit management of parallelism, particularly for large out-of-core
data sets.  Section~\ref{sec:lessons} summarizes some lessons learned
from the implementation and use of our library.

\section{MapReduce in MPI}
\label{sec:mr}

The basic datums stored and operated on by any MapReduce framweork are
key/value (KV) pairs.  In the MR-MPI library, individual keys or
values can be of any data type or length, or combinations of multiple
types (one integer, a string of characters, two integers and a double,
etc); they are simply treated as byte strings by the library.  A KV
pair always has a key; its value may be NULL.  A related data type is
the key/multivalue (KMV) pair, where all values associated with the
same key are collected and stored contiguously as a multivalue, which
is just a longer byte string with an associated vector of integer
lengths, one per value.

A typical MR-MPI program makes at least 3 calls to the MR-MPI library,
to perform a {\it map()}, {\it collate()}, and {\it reduce()}
operation.  In a {\it map()}, zero or more key/value pairs are
generated by each processor.  Typically, this is done using data read
in from files, but a {\it map()} may generate data itself or process
existing KV pairs to create new ones.  The KV pairs produced are
stored locally by each processor; a {\it map()} thus requires no
inter-processor communication.  Users call the library with a count of
tasks to perform and a pointer to a user function; the MR-MPI {\it
map()} operation invokes the user function multiple times as a
callback.  Depending on which variant of {\it map()} is called, the
user function may be passed a file name, a chunk of bytes from a large
file, or a task ID.  Options for assigning map tasks to processors.

The {\it collate()} operation (or data shuffle in Hadoop) identifies
unique keys and collects all the values associated with those keys to
create KMV pairs.  This is done in two stages, the first of which
requires communication, since KV pairs with the same key may be owned
by any processor.  Each processor hashes each of its keys to determine
the processor who ``owns'' it.  The $N$-byte length key is hashed into
a 32-bit value whose remainder modulo $P$ generates the owning
processor ID.  Alternatively, the user can provide a hash function
which converts the key into a processor ID.  Each processor then sends
each of its KV pairs to the owning processor.

After receiving new KV pairs, the second stage is an on-processor
computation, requiring no further communication.  Each processor
reorganizes its KV pairs into KMV pairs, one for each unique key it
owns.  This is done using a hash table, rather than a sort.  A sort
requires $\log(N)$ passes through the $N$ KV pairs.  the list of KMV
pairs can be created in 2 passes through the KV pairs, one to populate
the hash table with needed count and length information, and the
second pass to copy the key and value datums into the appropriate
location in a new KMV data structure.  Since the lookup of a key in a
well-formed hash table is a constant-time $O(1)$ operation, the cost
of the data reorganization is $O(N)$, rather than $N\log(N)$ for full
sort.  It also has the added benefit that the number of values in each
KMV pair is known, which is passed to the user function during a
reduce.  For some reduce operations, this is all the information a
reduce requires; thus the values need not be looped over.  This is not
the case for a Hadoop-style data reorg which sorts the values.  It
does not know a priori how many values are associated with the same
key without iterating over them.

Note that the first portion of the {\it collate()} operation involves
all-to-all communication (each processor sends and receives data from
every other processor) driven by a distributed hash table.  The
communication can either be done via a MPI\_Alltoall() library call,
or by a custom routine that aggregates messages and invokes pointwise
MPI\_Send() and MPI\_IRecv() calls.

The {\it reduce()} operation processes KMV pairs and can produce new
KV pairs for continued computation.  Each processor operates only on
the KMV pairs it owns; no communication is required.  As with the {\it
map()}, users call the library with a pointer to a user function.  The
MR-MPI {\it reduce()} operation invokes the user function, once for
each KMV pair.

Several related MapReduce operations are provided by the library.  For
example, the {\it compress()} operation combines on-processor KV pairs
to eliminate duplicate keys, forming new KV pairs with an aggregated
value.  The { \it convert()} operation turns a list of KV pairs into
KMV pairs, with one value per key.  The {collapse()} operation turns
$N$ KV pairs into one KMV pair, with the keys and values of the KV
pairs becoming $2N$ values of a single multivalue assigned to a new
key on the KMV.  The {\it gather()} operation collects KV pairs from
all processors to a subset of processors; it is useful for doing
output from a single or subset of processors.  Library calls for
sorting datums by key or value or within multivalues are also
provided.  These routines invoke the C-library quicksort() function to
compute the sorted ordering of the KV pairs (or values within a
multivalue), using a user-provided comparison function.  The KV pairs
(or values in a multivalue) are then copied one-by-one into a new data
structure in sorted order.

The purpose of providing an interface to various lo-level operations,
is that a user program can string them together in various ways to
produce interesting MapReduce algorithms.  For example, output from a
{\it reduce()} can serve as input to a subsequent {\it map()} or {\it
collate()}.  KV pairs from mutiple MapReduce objects can be combined
together to perform new sequences of {\it map()}, {\it collate()}, and
{\it reduce()} operations.

The above discussion assumed that the KV or KMV pairs stored by a
processor fit in its physical memory.  If this is the case, then we
refer to this as ``in-core'' processing and no disk files are written
or read by any of the processors, aside from initial input data if it
exists or final output data if it is generated.  Note that the
aggregate physical memory of large parallel machines can be multiple
Tbytes, which allows for large data sets to be processed in-core,
assuming the KV and KMV pairs remain evenly distributed across
processors throughout the sequence of MapReduce operations.  The use
of randomized hashing to assign keys to processors is designed for
such load-balancing.  What happens when data sets do not
fit in available memory, is the subject of the next section.

\section{Out-of-core Issues}

  what part needs to be out-of-core in MPI context
  each proc does it's own disk IO
  basic idea: modified data structures for KV and KMV
  map() operation
  aggreate() operation
  convert() operation
  reduce() operation

\section{Graph Algoriths in MapReduce}
\label{sec:graph}

We begin with a MapReduce procedure for creating large, sparse,
randomized graphs, since they are the input for the algorithms
discussed below.  R-MAT graphs ~\cite{RMAT} are recursively generated
graphs with power-law degree distributions.  They are commonly used to
represent web and social networks.  The user specifices 6 parameters
which define the graph: the number of vertices $N$ and edges $M$, and
4 parameters $a$, $b$, $c$, $d$ which sum to 1.0 and are discussed
below.  The algorithm in Figure \ref{fig:rmat} generates $N_z = MN$
unique non-zero entries in a sparse $NxN$ matrix $A$, where each entry
$A_{ij}$ represents an edge between graph vertices $(V_i,V_j)$.

\begin{figure}[htb]
 \begin{center}\fbox{\begin{minipage}{\textwidth} \begin{tabbing}
  xxxx\=xxxxxxxxxxxx\=xxxxxxxxxxxxxxxxxxxxxxxxxxxxxxxxxx\= \kill

$N_{\rm remain} = N_z$ \\
while $N_{\rm remain} > 0$: \\
\> 1 Map: \> Generate $N_{\rm remain}/P$ random edges on each processor \\
           \> \> output Key = $(V_i,V_j)$, Value = NULL \\
\> 1 Collate \\
\> 1 Reduce: \> Remove duplicate edges \\
              \> \> input Key = $(V_i,V_j)$, MultiValue = one or more NULLs \\
              \> \> output Key = $(V_i,V_j)$, Value = NULL \\
\> $N_{\rm remain} = N_z - N_{kv}$

  \end{tabbing}
 \end{minipage}}\end{center}

 \caption{MapReduce algorithm for R-MAT graph generation.}

 \label{fig:rmat}
\end{figure}

In the map() operation, each of $P$ processors generates a $1/P$
fraction of the desired edges.  A single random $i,j$ edge is computed
recursively as follows.  Pick a random quadrant of the $A$ matrix with
relative probabilities $a$, $b$, $c$, and $d$.  Treat the chosen
quadrant as a sub-matrix and select a random quadrant within it, in
the same manner.  Repeat this process $n$ times where $N = 2^n$.  At
the end of the recursion, the final ``quadrant'' is non-zero matrix
element $A_{ij}$.

The map() will often generate some small number of duplicate edges.
The collate() and reduce() operations remove the duplicates.  The
entire map-collate-reduce sequence is repeated until the number of
resulting key/value pairs $N_{kv}$ equals $N_z$.  For reasonably
sparse graphs this typically takes only a few iterations.

Note that the degree distribution of vertices in the graph depends on
the choice of parameters $a$, $b$, $c$, $d$.  If one of the four
values is larger than the other 3, a highly skewed distribution
results.  Variants of the above algorithm can be used when $N$ is not
a power-of-two, to generate graphs with weighted edges (assign a
numeric value to the $A_{ij}$ edge), graphs without self edges
(require $i != j$), or graphs with undirected edges (require $i < $j).
Or the general R-MAT matrix can be further processed by MapReduce
operations to meet these requirements.

The PageRank algorithm assings a relative numeric rank to each
vertex in a graph.

It models the web as a directed graph $G(V,E)$, with each vertex $v
\in V$ representing a web page and each edge $e_{ij} \in E$
representing a hyperlink from $v_i$ to $v_j$.  The probability of
moving from $v_i$ to another vertex $v_j$ is $\alpha/d_{out}(v_i) +
(1-\alpha)/|V|$, where $\alpha$ is a user-defined parameter (usually
0.8-0.9), $d_{out}(v)$ is the outdegree of vertex $v$, and $|V|$ is
the cardinality of $V$.  The first term represents the probability of
following a given link on page $v_i$; the second represents the
probability of moving to a random page.  For pages with no outlinks,
the first term is $\alpha/|V|$, indicating equal likelihood to move to
any other page.  Equivalently, the graph can be represented by a
matrix $A$~\cite{LangvilleMeyer05a}, with matrix entries $A_{ij} =
\alpha/d_{out}(v_i)$ if vertex $v_i$ links to $v_j$.  The PageRank
algorithm, then, is simply a power-method iteration in which the
dominating computation is matrix-vector multiplication $A^T x=y$,
where $x$ is the PageRank vector from the previous iteration.

The algorithm in Figure \ref{fig:pr} performs these
iterations.

\begin{figure}[htb]
 \begin{center}\fbox{\begin{minipage}{\textwidth} \begin{tabbing}
  xxxx\=xxxxxxxxxxxx\=xxxxxxxxxxxxxxxxxxxxxxxxxxxxxxxxxx\= \kill

$N_{\rm remain} = N_z$ \\
while $N_{\rm remain} > 0$: \\
\> Map(): \> {\bf Generate} $N_{\rm remain}/P$ random edges on each processor \\
           \> \> output Key = $(V_i,V_j)$, Value = NULL \\
\> Collate() \\
\> Reduce(): \> {\bf Remove} duplicate edges \\
              \> \> input Key = $(V_i,V_j)$, MultiValue = one or more NULLs \\
              \> \> output Key = $(V_i,V_j)$, Value = NULL \\
\> $N_{\rm remain} = N_z - N_{kv}$

  \end{tabbing}
 \end{minipage}}\end{center}

 \caption{MapReduce algorithm for PageRank vertex ranking.}

 \label{fig:pr}
\end{figure}

The MapReduce implementation performs two {\it map()} operations to
initialize the graph matrix $A$ and PageRank vector $x$.  A {\it
collate()} operation gathers all row entries $a_{ij}$ with their
associated $x_i$ entry, and a {\it reduce()} computes $a_{ij} x_i$.  A
second {\it collate()} gathers, for each $j$, all contributions to the
column sum $\sum a_{ij} x_i$, which is computed by a second {\it
reduce()}.  MPI\_Allreduce calls are used to compute global norms and
residuals.

A triangle in a graph is any triplet of vertices $(V_i,V_j,V_k)$ where
the edges $(V_i,V_j), (V_j,V_k), (V_i,V_k)$ exist.  Figure
\ref{fig:tri} outlines a MapReduce algorithm that enumerates all
triangles, assuming an input graph of undirected edges $(V_i,V_j)$
where $V_i < V_j$ for every edge, i.e. an upper-triangular R-MAT
matrix.  This exposition follows the triangle-finding algorithm
presented in \cite{Cohen}.

\begin{figure}[htb]
 \begin{center}\fbox{\begin{minipage}{\textwidth} \begin{tabbing}
  xxxxxxxxxxxx\=xxx\=xxx\=xxx\=xxxxxxxxxxxxxxxxxxxxxxxxxxxx\= \kill

1 Copy: \> $G_0$ = copy of edge KV pairs from input graph \\
1 Map: \> Convert edges to vertices \\
           \> \> input Key = $(V_i,V_j)$, Value = NULL \\
           \> \> output Key = $V_i$, Value = $V_j$ \\
           \> \> output Key = $V_j$, Value = $V_i$ \\
1 Collate \\
1 Reduce: \> Add first degree to one vertex in edge \\
              \> \> input Key = $V_i$, MultiValue = $(V_j, V_k, ...)$ \\
	      \> \> for each V in MultiValue: \\
              \> \> \> if $V_i < V$: output Key = $(V_i,V)$, Value = $(D_i,0)$ \\
              \> \> \> else: output Key = $(V,V_i)$, Value = $(0,D_i)$ \\
2 Collate \\
2 Reduce: \> Add second degree to other vertex in edge \\
              \> \> input Key = $(V_i,V_j)$, MultiValue = $((D_i,0),(0,D_j))$ \\
              \> \> output Key = $(V_i,V_j)$, Value = $(D_i,D_j)$ with $V_i < V_j$ \\
3 Map: \> Low degree vertex emits edges \\
           \> \> if $D_i < D_j$: output Key = $V_i$, Value = $V_j$ \\
           \> \> else if $D_j < D_i$: output Key = $V_j$, Value = $V_i$ \\
           \> \> else: output Key = $V_i$, Value = $V_j$ \\
3 Collate \\
3 Reduce: \> Emit angles of each vertex \\
              \> \> input Key = $V_i$, MultiValue = $(V_j, V_k, ...)$ \\
	      \> \> for each $V_1$ in MultiValue: \\
	      \> \> \> for each $V_2$ beyond $V_1$ in MultiValue: \\
	      \> \> \> \> if $V_1 < V_2$: output Key = $(V_1,V_2)$, Value = $V_i$ \\
	      \> \> \> \> else: output Key = $(V_2,V_1)$, Value = $V_i$ \\
4 Add: \> Add $G_0$ edge KV pairs to angle KV pairs \\
4 Collate \\
4 Reduce: \> Emit triangles \\
              \> \> input Key = $(V_i,V_j)$, MultiValue = $(V_k,V_l,NULL,V_m,...)$ \\
              \> \> if NULL exists in MultiValue: \\
	      \> \> \> for each non-NULL V in MultiValue: \\
	      \> \> \> \> output Key = $(V_i,V_j,V)$, Value = NULL

  \end{tabbing}
 \end{minipage}}\end{center}

 \caption{MapReduce algorithm for triangle enumeration.}

 \label{fig:tri}
\end{figure}

The initial step is to store a copy of the graph edges as key/value
(KV) pairs in a auxiliary MapReduce object $G_0$, for use later in the
algorithm.  The first {\it map()} operation converts edge keys to
vertex keys with edge values.  After the {\it collate()}, each vertex
has a list of vertices it is connected to; the first {\it reduce()}
can thus flag one vertex $V_i$ in each edge with a degree count $D_i$.
The second {\it collate()} and {\it reduce()} assign a degree count
$D_j$ to the other vertex in each edge.  In the third {\it map()},
only the lower-degree vertex in each edge emits its edges as key/value
(KV) pairs.  The task of the third {\it reduce()} is to emit
``angles'' for each of these low-degree vertices.  An ``angle'' is a
root vertex $V_i$, with two edges to vertices $V_1$ and $V_2$, i.e. a
triangle without the third edge $(V_1,V_2)$.  The {\it reduce()} emits
a list of all angles of vertex $V_i$, by a double loop over the edges
of $V_i$.  Note that the aggregate volume of KV pairs emitted at this
stage is minimized by having only the low-degree vertex in each edge
generate angles.

In stage 4, the KV pairs in the original graph $G_0$ are added to the
current working set of KV pairs.  The KV pairs in $G_0$ are edges that
complete triangles for the angle KV pairs just generated.  After the
fourth {\it collate()}, a pair of vertices $(V_i,V_j)$ is the key, and
the multivalue is the list of all root vertices in angles that contain
$V_i$ and $V_j$.  If the multivalue also contains a NULL, contributed
by $G_0$, then there is a $(V_i,V_j)$ edge in the graph.  Thus all
vertices in the multivalue are roots of angles which are complete
triangles and can be emitted as a triplet key.

A connected component of a graph is a set of vertices where all pairs
of vertices in the set are connected by a path of edges.  A sparse
graph may contain many such components.  Figure \ref{fig:cc} outlines
a MapReduce algorithm that labels each vertex in a graph with a
component ID.  All vertices in the same component are labelled with
the same ID, which is the ID of a vertex in the component.  We assume
an input graph of undirected edges $(V_i,V_j)$.  This exposition also
follows the connected-component algorithm presented in \cite{Cohen},
with the addition of logic that load-balances data across processors
when one or a few giant components exist in the graph.

\begin{figure}[htb]
 \begin{center}\fbox{\begin{minipage}{\textwidth} \begin{tabbing}
  xxxx\=xxxxxxxxxxx\=xxxx\=xxxx\=xxxxxxxxxxxxxxxxxxxxxxxxxxxxxxxxxx\= \kill

Iterate: \\
\> 1 Map: \> Convert edges to vertices \\
    \> \> \> input Key = $E_{ij}$, Value = NULL \\
    \> \> \> output Key = $V_i$, Value = $E_{ij}$ \\
    \> \> \> output Key = $V_j$, Value = $E_{ij}$ \\
\> 1 Add: \> Zone assignment of each vertex \\
    \> \> \> output Key = $V_i$, Value = $Z_i$ \\
\> 1 Collate: \> Vertex as key \\
\> 1 Reduce: \> Emit edges of each vertex with zone of vertex \\
       \> \> \> input Key = $V_i$, MultiValue = $E E E E ... Z$ \\
       \> \> \> for each $E$ in MultiValue: \\
      \> \> \> \> output Key = $E_{ij}$, Value = $Z_i$ \\

\> 2 Collate: \> Edge as key \\
\> 2 Reduce: \> Emit zone re-assignments \\
       \> \> \> input Key = $E_{ij}$, MultiValue = $Z_i Z_j$ \\
       \> \> \> $Z_{winner}$ = min($Z_i$,$Z_j$); $Z_{loser}$ = max($Z_i$,$Z_j$) \\
       \> \> \> if $Z_i$ and $Z_j$ are different: \\
      \> \> \> \> output Key = $Z_{loser}$, Value = $Z_{winner}$ \\
\> 2 Exit: \> if no output by Reduce 2 \\
\> 3 Map: \> Invert vertex/zone pairs \\
    \> \> \> input Key = $V_i$, Value = $Z_i$ \\
    \> \> \> if $Z_i$ is not partitioned: \\
   \> \> \> \> output Key = $Z_i$, Value = $V_i$ \\
    \> \> \> else: \\
   \> \> \> \> output Key = $Z_i^+$, Value = $V_i$ for a random processor \\
\> 3 Add: \> Changed zones ($Z_i$,$Z_{winner}$) \\
    \> \> \> if $Z_i$ is not partitioned: \\
   \> \> \> \> output Key = $Z_i$, Value = $Z_{winner}$ \\
    \> \> \> else: \\
   \> \> \> \> output Key = $Z_i^+$, Value = $Z_{winner}$ for every processor \\
   \> \> \> \> output Key = $Z_i$, Value = $Z_{winner}$ \\
\> 3 Collate: \> Zone ID as key \\
\> 3 Reduce: \> Emit new zone assignment of each vertex \\
       \> \> \> input Key = $Z_i$ or $Z_i^+$, MultiValue = $V V V V ... Z Z Z ...$ \\
       \> \> \> $Z_{new}$ = min($Z_i$ or $Z_i^+$,Z,Z,Z,...) \\
       \> \> \> partition $Z_{new}$ if number of $V$ > threshhold \\
       \> \> \> for each $V$ in MultiValue: \\
      \> \> \> \> output Key = $V_i$, Value = $Z_{new}$

  \end{tabbing}
 \end{minipage}}\end{center}

 \caption{MapReduce algorithm for connected component labelling.}

 \label{fig:cc}
\end{figure}

The algorithm begins (before the iteration loop) by assigning each
vertex to its own component or ``zone'', so that $Z_i$ = $V_i$.  Each
iteration will grow the zones, one layer of neighbors at a time.  As
zones collide due to shared edges, a winner is chosen (the smaller
zone ID), and vertices in the losing zone are reassigned to the
winning zone.  When the iterations complete, each zone will have
become a fully connected component.  The algorithm thus finds all
connected components in the graph simultaneously.  The number of
iterations required depends on the largest diameter of any component
in the graph.

The first {\it map()} operation emits the vertices in each edge as
keys, with the edge as a value.  The current zone assignment of each
vertex is added to the set of key/value pairs.  The first collate()
operation collects all the edges of a vertex and its zone assignment
together in one multi-value.  The first reduce operation then re-emits
each edge, tagged by the zone assignment of one of its vertices.

Since each edge was emitted twice, the second collate operation
collects the zone assignments for its two vertices together.  If the
two zone IDs are different, the second reduce operation chooses a
winner (the min of the 2 IDs), and emits the loser ID as a key, with
the winning ID as a value.  If no zone ID changes are emitted, the
algorithm is finished, and the iteration exits.

The third map() operation inverts the vertex/zone key/value pairs to
become zone/vertex pairs.  The third add operation adds the changing
zone assignments to the set of key/value pairs.  The third collate()
can then collects all the vertices of a zone and zero or more
reassignments for the zone ID.  Since a zone could collide with
multiple other zones on the same iteration due to shared edges, the
new zone ID becomes the minimum ID of any of the neighboring zones.
If no zone reassignment value appears in the multi-value, the zone ID
is unchanged.  The final reduce() a key/value pair for each vertex in
the zone, with the vertex as a key and the new zone ID as a value.

Note that if a graph has only a few components, then the third collate
operation, which keys on the zone ID, may generate a few very large
key/multi-value (KMV) pairs.  For example, if the graph is fully
connected, then on the last iteration, a single KMV pair will contain
all vertices in the graph and be assigned to one processor.  This
imbalance in memory and computational work can lead to poor parallel
performance of the overall algorithm.  To counter this effect, the
various operations of stage 3 include extra logic.  The idea is to
partition zones whose vertex count exceeds a user-defined threshhold
into $P$ sub-zones, where $P$ is the number of processors.  The 64-bit
integer that stores the zone ID also stores a bit flag indicating the
zone has been partitioned and a set of bits that encode the processor
ID.

During the third map() operation, if the zone has been partitioned,
then the vertex is assigned to a random processor and the processor ID
bits are added to the zone ID, as indicated by the $Z_i^+$ notation in
Figure \ref{fig:cc}.  Likewise, if $Z_i$ has been partitioned, the
third add() operation emits the zone ID change key/value pair
($Z_i$,$Z_{winner}$) as ($Z_i^+$,$Z_{winner}$).  In this case
($Z_i$,$Z_{winner}$), is emitted not once, but $P+1$ times, once for
each processor, and once as if $Z_i$ had not been partitioned (for a
reason discussed below).

This additional partitioning logic means that the third collate()
operation, which keys on zone IDs, some of which now include processor
bits, will collect only a $1/P$ subset of the vertices in large zones
onto each processor.  But the multivalue on each processor contains
all the zone-reassignments relevant to the unpartitioned zone.  This
allows the third reduce operation to change the zone ID (if necessary)
in a consistent manner across all $P$ multi-values that contain the
zone's vertices.  When the reduce() operation emits new zone
assignments for each vertex, the zone retains its partitioned status,
and the partition bit is also explicitly set if the vertex count
exceeds the threshhold for the first time.

Note that this logic does not guarantee that the partition bits of the
zone IDs for all the vertices in a single zone will be set
consistently on a given iteration.  For example, an unpartitioned zone
with a small ID may consume a partitioned zone.  The vertices from the
partitioned zone will retain their partitioned status, but the
original vertices in the small zone may not set the partition bit of
their zone IDs.  On subsequent iterations, the third add() operation
emits $P+1$ copies of new zone reassignments for both partitioned and
unpartioned zone IDs, to insure all vertices in the zone will know the
reassignment information.

Discussion of Luby algorithm for maximally independent sets.

footnote on NP-complete for maximal; MapReduce is unlikely to
help with that issue.

\begin{figure}[htb]
 \begin{center}\fbox{\begin{minipage}{\textwidth} \begin{tabbing}
  xxxx\=xxxxxxxxxxxx\=xxxxxxxxxxxxxxxxxxxxxxxxxxxxxxxxxx\= \kill

$N_{\rm remain} = N_z$ \\
while $N_{\rm remain} > 0$: \\
\> Map(): \> {\bf Generate} $N_{\rm remain}/P$ random edges on each processor \\
           \> \> output: Key = $(V_i,V_j)$, Value = NULL \\
\> Collate() \\
\> Reduce(): \> {\bf Remove} duplicate edges \\
              \> \> input: Key = $(V_i,V_j)$, MultiValue = one or more NULLs \\
              \> \> output: Key = $(V_i,V_j)$, Value = NULL \\
\> $N_{\rm remain} = N_z - N_{kv}$

  \end{tabbing}
 \end{minipage}}\end{center}

 \caption{MapReduce algorithm for R-MAT graog generation.}

 \label{fig:rmat}
\end{figure}

Discussion of SSSP.

Discuss new optimization in SSSP?

\begin{figure}[htb]
 \begin{center}\fbox{\begin{minipage}{\textwidth} \begin{tabbing}
  xxxx\=xxxxxxxxxxxx\=xxxxxxxxxxxxxxxxxxxxxxxxxxxxxxxxxx\= \kill

$N_{\rm remain} = N_z$ \\
while $N_{\rm remain} > 0$: \\
\> Map(): \> {\bf Generate} $N_{\rm remain}/P$ random edges on each processor \\
           \> \> output: Key = $(V_i,V_j)$, Value = NULL \\
\> Collate() \\
\> Reduce(): \> {\bf Remove} duplicate edges \\
              \> \> input: Key = $(V_i,V_j)$, MultiValue = one or more NULLs \\
              \> \> output: Key = $(V_i,V_j)$, Value = NULL \\
\> $N_{\rm remain} = N_z - N_{kv}$

  \end{tabbing}
 \end{minipage}}\end{center}

 \caption{MapReduce algorithm for single-source shortest path (SSSP)
 determination.}

 \label{fig:rmat}
\end{figure}

\section{Performance Results}
\label{sec:results}

In this section, we present performance results for the MapReduce
graph algorithms of the preceeding section, implemented as small C++
programs calling our MR-MPI library.  The benchmarks were run on a
medium-sized Linux cluster of 2 GHz dual-core AMD Opteron processors
connected via a Myrinet network.  Most importantly for MR-MPI, each
node of the cluster has one local disk, which is used for out-of-core
operations in MR-MPI.  To avoid contention for disk I/O, we ran all
experiments with one MPI process per node.  For comparisons with other
implementations, we used either the same cluster or, where noted,
Sandia's Cray XMT, a multi-threaded parallel computer with 500 {MHz}
processors and a 3D-Torus network.

We ran each of the algorithms on three R-MAT graphs of different
sizes, each on a varying number of processors.  Details of the input
data are shown in Table~\ref{t:rmats}.  The {\it small} problem
(around 8M edges) can typically be run on a single processor without
incurring out-of-core operations.  The {\it medium} problem (around
134M edges) can be run on a single processor with out-of-core
operations; larger processor configurations, however, do not
necessarily require out-of-core operations.  The {\it large} problem
(around 2B edges) requires out-of-core operations on our 64-node
cluster.

%The {\it x-large} problem 
%(around 34B edges) requires most of the machine to run.

All data sets used R-MAT parameters $(a, b, c, d) = (0.57, 0.19, 0.19,
0.05)$ and generated 8 edges per vertex (on average).  These values
create a highly skewed degree distribution, as indicated by the
maximumn vertex degree in Table~\ref{t:rmats}.

\begin{table}
\begin{center}
\begin{tabular}{|l|c|c|c|c|c|c|c|}
\hline
Data & \# of    & \# of & Maximum \\
Set  & vertices & edges & vertex degree\\
\hline
RMAT-20 (small)   &$2^{20} \approx 1M$ & $2^{23} \approx 8M$ &  $\approx 24K$ \\
RMAT-24 (medium)  &$2^{24} \approx 17M$ & $2^{27} \approx 134M$ &  $\approx 147K$ \\
RMAT-28 (large)   &$2^{28} \approx 268M$ & $2^{31} \approx 2B$& $\approx 880K$ \\
%RMAT-32 (x-large) &$2^{32} \approx 4B$ & $2^{35} \approx 34B$ &   \\
\hline
\end{tabular}
\caption{Characteristics of R-MAT input data for graph algorithm
benchmarks.}
\label{t:rmats}
\end{center}
\end{table}

The resulting timings give a sense of the inherent scalability of the
MapReduce algorithms as graph size grows on a fixed number of
processors, and of the parallel scalability for computing on a graph
of fixed size on a growing number of processors.  Where available, we
compare the MapReduce algorithm with other parallel implementations,
including more traditional distributed-memory algorithms and
multi-threaded algorithms in the Multi-Threaded Graph Library
(MTGL)~\cite{MTGL} on the Cray XMT.  We compute parallel efficiency on
$P$ processors as $({time}_{M} \times M) / ({time}_P \times P)$ where
$M$ is the smallest number of processors on which the experiment was
run, and ${time}_I$ is the execution time required on $I$ processors.

\subsection{R-MAT generation results}

In Figure~\ref{f:rmat}, we show the scalability of the R-MAT
generation algorithm (Figure~\ref{fig:rmat2}) for RMAT-20, RMAT-24 and
RMAT-28.  For RMAT-20 and RMAT-24, superlinear speed-up is shown.
This speed-up is due to the decreased amount of file I/O needed with
greater numbers of processors; with a larger total memory, a
fixed-size problem requires less I/O with more processors.  For
RMAT-28, which requires significant out-of-core operations, parallel
efficiency ranges from 52\% to 97\%.  Comparisons between the enhanced
algorithm (Figure~\ref{fig:rmat2}) and the original algorithm
(Figure~\ref{fig:rmat}) showed approximately 10\% reduction in
execution time for the enhanced algorithm.

%Also in Figure~\ref{f:rmat}, we compare our MR-MPI implementation with an 
%R-MAT generator in MTGL on the Cray XMT.  In the MTGL implementation, 
%threads generate edges independently.  A global hash table is used to remove
%duplicate edges.  Our MR-MPI implementation runs more quickly than the MTGL
%implementation.  The differences in execution time are greatest on 64
%processors, where MR-MPI needs the least file I/O for out-of-core operations.

\begin{figure}[htb]
\includegraphics[width=\textwidth]{fig_rmat.pdf}
\caption{Performance of the MR-MPI R-MAT generation algorithm (Figure~\ref{fig:rmat2}) 
%and an MTGL implementation.
}
\label{f:rmat}
\end{figure}

\subsection{PageRank results}
\label{subsec:results_pagerank}

In Figure~\ref{f:pr}, we show the performance of the MR-MPI PageRank
algorithm (Figure~\ref{fig:pr2}) compared to a distributed-memory
matrix-based implementation using the linear algebra toolkit
Trilinos~\cite{Trilinos-Overview}.  The matrix-based
distributed-memory implementation of PageRank uses Trilinos Epetra
matrix/vector classes to represent the graph and PageRank vector.
Rows of matrix $A$ and the associated entries of the PageRank vector
$x$ are uniquely assigned to processors; a random permutation of the
input matrix effectively load balances the non-zero matrix entries
across processors.  Interprocessor communication gathers $x$ values
for matrix-vector multiplication and sums partial products into the
$y$ vector.  Most communication is point-to-point communication, but
some global communication is needed for computing residuals and norms
of $x$ and $y$.

Figure~\ref{f:pr} shows the execution time per PageRank iteration for
R-MAT matrices RMAT-20, RMAT-24 and RMAT-28.  Converging the PageRank
iterations to tolerance $0.002$ requires five or six iterations.
Several R-MAT matrices of each size were generated for the
experiments; the average time over the matrices is reported here.  The
Trilinos implementations show near-perfect strong scaling for RMAT-20
and RMAT-24.  The MR-MPI implementations also demonstrate good strong
scaling.  However, MR-MPI's execution time is at least an order of
magnitude greater than the Trilinos implementation.  This result is
due to two factors: ({\it i}) a higher volume of communication in the
MapReduce implementation (which comunicates one datum per graph edge
in each iteration), compared to the Trilinos implementation which only
communicates a volume of data proportional to the number of vertices,
and ({\it ii}) out-of-core operations in MR-MPI were performed because
of the page-size restriction, while all Trilinos operations were
performed in-core.  The benefit of MR-MPI's out-of-core implementation
is seen, however, with the RMAT-28 data set, which could be solved on
smaller processor sets than the Trilinos implementation.  For these
experiments, the Trilinos implementation required 64 processors for
the RMAT-28 data set, since it will not operate on matrices that
exceed the aggregate memory of the processors.

\begin{figure}[htb]
\includegraphics[width=\textwidth]{fig_pagerank.pdf}
\caption{Comparison of PageRank implementations using 
MR-MPI (Figure~\ref{fig:pr2}) and 
Trilinos' matrix/vector classes on R-MAT data sets.}
\label{f:pr}
\end{figure}

\subsection{Triangle finding results}

In Figure~\ref{f:tri}, we show the performance of the triangle finding
algorithm (Figure~\ref{fig:tri}).  Execution times for this algorithm
were too large to allow the problem to be easily run with RMAT-28 on
our cluster.  Parallel efficiencies for RMAT-20 ranged from 80\% to
140\%; for RMAT-24, they ranged from 50\% to 120\%.

%Comparisons with an MTGL implementation on the 
%Cray XMT give mixed results.  When MR-MPI uses the least file I/O for 
%out-of-core operations (e.g., RMAT-20 on 32 and 64 processors), it 
%outperforms the MTGL implementation.  However, when significant out-of-core
%operations are needed (e.g., for RMAT-24), the MTGL implementation is faster.

\begin{figure}[htb]
\includegraphics[width=\textwidth]{fig_tri.pdf}
\caption{Performance of the MR-MPI triangle-finding algorithm
(Figure~\ref{fig:tri}).}
\label{f:tri}
\end{figure}

\subsection{Connected Components}

In Figure~\ref{fig:cc}, we show the performance of the connected
component identification algorithm (Figure~\ref{fig:cc}).  A
comparison is made with a hybrid ``Giant Connected Component''
implementation using both Trilinos and MR-MPI.  Power law graphs often
have one or two very large components and many very small components.
The hybrid algorithm exploits this feature of the data by using
inexpensive breadth-first search (BFS) from the vertex with highest
degree to identify the largest components, followed by a more
expensive algorithm to identify the small components in the remainder
of the graph.  In our hybrid implementation, we intially perform
matrix-vector multiplications in Trilinos to perform a BFS, finding
components that include (in total) 70\% or more of the vertices.  We
then apply our MR-MPI algorithm to the remaining graph to identify the
small components.  This hybrid approach is quite effective, reducing
the execution time to identify all components of RMAT-20 and RMAT-24
by 80-99\%, as shown in Figure~\ref{f:cc}.  The benefit of MR-MPI is
seen, however, for RMAT-28, where the graph is too large to fit into
memory, and thus Trilinos cannot perform the initial BFS for our
hybrid algorithm.

\begin{figure}[htb]
\includegraphics[width=\textwidth]{fig_cc.pdf}
\caption{Performance of the MR-MPI connected components algorithm (Figure~\ref{fig:cc}) compared with a hybrid ``Giant Connected Component'' algorithm based
on Trilinos.}
\label{f:cc}
\end{figure}

\subsection{Maximally independent set results}

The execution times for the maximal independent set algorithm
(Figure~\ref{fig:luby}) are shown in Figure~\ref{f:luby}.  Like the
R-MAT generation results, superlinear speed-up of the algorithm occurs
for RMAT-20 and RMAT-24, as more of the graph fits into processor
memory and less file I/O is needed.  For RMAT-28, the algorithm
requires significant out-of-core operations. In this case, parallel
efficiency is nearly perfect going from 8 to 64 processors.

\begin{figure}[htb]
\includegraphics[width=\textwidth]{fig_luby.pdf}
\caption{Performance of the MR-MPI maximal independent set algorithm (Figure~\ref{fig:luby}).}
\label{f:luby}
\end{figure}

\subsection{Single-source shortest path results}

The execution times for the single-source shortest path algorithms
(Figures~\ref{fig:sssp} and \ref{fig:sssp2}) are shown in
Figure~\ref{f:sssp}.  The results show the benefit of using the
enhanced algorithm, providing at least a 17\% (and often greater)
reduction in execution time due to reduced communication; only the
updated distances are communicated throughout most of the enhanced
algorithm.  However, the execution times are still large compared to
multi-threaded implementations; for example, Madduri et
al.~\cite{Madduri07} report execution times of only 11 seconds on 40
processors for a multithreaded implementation on a Cray MTA-2 (the
predecessor of the XMT) on graphs with one billion edges.

\begin{figure}[htb]
\includegraphics[width=\textwidth]{fig_sssp.pdf}
\caption{Execution times for SSSP using MR-MPI with R-MAT matrices.  
Both the original algorithm (Figure~\ref{fig:sssp}) and the enhanced 
algorithm (Figure~\ref{fig:sssp2}) are included.}
\label{f:sssp}
\end{figure}

To compare our MR-MPI implementation with a wider set of algorithms,
we performed experiments comparing MR-MPI, Hadoop, PBGL (Parallel
Boost Graph Library)~\cite{PBGL} and a multi-threaded implementation
on the Cray XMT using two web graphs: {WebGraphA} with 13.2M vertices
and 31.9M edges, and {WebGraphB} with 187.6M vertices and 531.9M
edges.  In Figure~\ref{f:ssspA}, we show execution times for the SSSP
algorithm using {WebGraphA}.  Like our MR-MPI implementation, the
Hadoop implementation is a Bellman-Ford-style~\cite{Bellman58,Ford62}
algorithm.  The XMT and PBGL implementations are based on
delta-stepping~\cite{MeyerSanders98}, and do not require full
iterations over the entire edge list to advance a breadth-first
search.  We observe that the MR-MPI implementation runs in less time
than the Hadoop implementation, but requires significantly more time
than the XMT and PBGL implementations.  In experiments with
{WebGraphB}, the benefit of the enhanced algorithm
(Figure~\ref{fig:sssp2}) is clearly shown, with a 40\% reduction in
execution time compared to the original SSSP algorithm
(Figure~\ref{fig:sssp}).  But the Bellman-Ford-style iterations are
especially harmful to the MR-MPI and Hadoop implementations for
{WebGraphB}, which required 110 iterations to complete; execution
times for this data set are shown in Table~\ref{t:ssspB}.

\begin{figure}[htb]
\includegraphics[width=\textwidth]{fig_ssspA.pdf}
\caption{SSSP using MR-MPI for WebGraphA with
13.2M vertices and 31.9M edges.  Runtimes using Hadoop and PBGL
are also shown.}
\label{f:ssspA}
\end{figure}

\begin{table}
\begin{center}
\begin{tabular}{|l|c|r|}
\hline
Implementation & Number of Processes & SSSP Execution Time \\
\hline
Hadoop & 48  & 38,925 secs.\\
MR-MPI original (Figure~\ref{fig:sssp}) & 48 &  13,505 secs.\\
MR-MPI enhanced (Figure~\ref{fig:sssp2}) & 48 &  8,031 secs.\\
XMT/C  & 32 &  37 secs.\\
%Out-of-core MR-MPI & 96 &   8,358 secs.\\
%Out-of-core MR-MPI & 64 &  12,882 secs.\\
%Out-of-core MR-MPI & 100 &  6,280 secs.\\
\hline
\end{tabular}
\caption{Execution times for SSSP with {WebGraphB}.}
\label{t:ssspB}
\end{center}
\end{table}

\subsection{Scalability to large numbers of processors}

Finally, we demonstrate the scalability of our MR-MPI library to large
numbers of processors.  The library was used on Sandia's Redstorm and
Thunderbird parallel computers.  Redstorm is a large Cray XT3 with 2+
{GHz} dual/quad-core AMD Opteron processors and a custom interconnect
providing 9.6 GB/s of interprocessor bandwidth.  Thunderbird is a
large Linux cluster with 3.6 {GHz} dual-core Intel EM64T processors
connected by an Infiniband network.  Because these systems do not have
local disks for each processor, we selected a data set and page sizes
that fit in memory, so out-of-core operations were not needed.  For
these experiments, we used an R-MAT data set with with $2^{25}$
vertices and $2^{28}$ edges, with parameters given in
Table~\ref{t:rmat}.  We ran both the PageRank and Connected Components
algorithms.

\begin{table}
\begin{center}
\begin{tabular}{|l|c|c|c|c|c|c|c|}
\hline
Data & R-MAT  & R-MAT  & R-MAT  & R-MAT  & \# of    & \# of & Maximum \\
Set  & a      & b      & c      & d      & vertices & edges & vertex degree\\
\hline
nice  & 0.45 & 0.15 & 0.15 & 0.25 & $2^{25}$ & $2^{28}$ & 1108 \\
nasty & 0.57 & 0.19 & 0.19 & 0.05 & $2^{25}$ & $2^{28}$ & 230,207\\
\hline
\end{tabular}
\caption{Characteristics of R-MAT input data for PageRank and Connected
Components scalability experiments.}
\label{t:rmat}
\end{center}
\end{table}

Figure~\ref{f:prbig}, shows the performance of the various PageRank
implementations on distributed memory and multi-threaded
architectures.  The MR-MPI and Trilinos implementations are described
in Sections~\ref{subsec:graph_pagerank}
and~\ref{subsec:results_pagerank}, respectively.  In the
multi-threaded MTGL implementation, rank propagates via adjacency list
traversal in a compressed sparse-row data structure.  To maintain its
scalability, code must be written so that a single thread spawns the
loop that processes all in-neighbors of a given vertex; this detail
enables the compiler to generate hotspot-free code.

The MR-MPI implementation demonstrated good scalability up to 1024
processors; however, as before, it required an order-of-magnitude more
execution time than the matrix-based implementations on Redstorm.  The
distributed memory matrix-based implementations are competitive with
the multi-threaded implementation in MTGL on the Cray XMT.

\begin{figure}[htb]
\includegraphics[width=\textwidth]{fig_pagerank_big.pdf}
\caption{Scalability comparison of PageRank using MapReduce (MR-MPI),
matrix-based (Trilinos), and multi-threaded (MTGL) implementation on
the R-MAT data sets in Table~\ref{t:rmat}.}
\label{f:prbig}
\end{figure}

Similar results were obtained for the Connected Components algorithm,
as shown in Figure~\ref{f:ccbig}.  As with PageRank, the MR-MPI
implementation showed good scalability up to 1024 processors, but
required significantly more time than the hybrid algorithm using
Trilinos and MR-MPI or the MTGL algorithm.

\begin{figure}[htb]
\includegraphics[width=\textwidth]{fig_cc_big.pdf}
\caption{Scalability comparison of Connected Components algorithms using 
MapReduce (MR-MPI),
matrix-based/MapReduce hybrid (Trilinos/MR-MPI), and MTGL implementations
on the R-MAT data sets in Table~\ref{t:rmat}.}
\label{f:ccbig}
\end{figure}


\section{Acknowledgements}
\label{sec:thanks}

We thank the following individuals for their contributions to this
paper: Greg Bayer and Todd Plantenga (Sandia) for explaining Hadoop
concepts to us, and for the Hadoop implementations and timings of
Section \ref{sec:results}; Jon Cohen (NSA) for fruitful discussions
about his MapReduce graph algorithms \cite{Cohen}; Brian Barrett
(Sandia) for the PBGL results of Section \ref{sec:results}; Jon Berry
(Sandia) for the MTGL results of Section \ref{sec:results}, and for
his overall support of this work and many useful discussions.

Sandia National Laboratories is a multi-program laboratory operated by
Sandia Corporation, a wholly owned subsidiary of Lockheed Martin
company, for the U.S. Department of Energy's National Nuclear Security
Administration under contract DE-AC04-94AL85000.


\bibliographystyle{aip}
\bibliography{paper}

\end{document}
